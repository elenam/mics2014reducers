\documentclass{beamer}
\usetheme{Montpellier}
\usecolortheme{dolphin}

\usepackage{graphicx} %For jpg figure inclusion
\usepackage{times} %For typeface
\usepackage{epsfig}
\usepackage{color} %For Comments
\usepackage[all]{xy}
\usepackage{float}
\usepackage{subfigure} 
\usepackage{hyperref}
\usepackage{url}
\usepackage{parskip}

%% Elena's favorite green (thanks, Fernando!)
\definecolor{ForestGreen}{RGB}{34,139,34}
%% Joe's Color.
\definecolor{JoesGold}{RGB}{204,102,0}
%% Henry's Color.
\definecolor{Teal}{RGB}{2,132,130}
% Uncomment this if you want to show work-in-progress comments
\newcommand{\comment}[1]{{\bf \tt  {#1}}}
% Uncomment this if you don't want to show comments
%\newcommand{\comment}[1]{}
\newcommand{\emcomment}[1]{\textcolor{ForestGreen}{\comment{Elena: {#1}}}}
\newcommand{\joecomment}[1]{\textcolor{JoesGold}{\comment{Joe: {#1}}}}
\newcommand{\hfcomment}[1]{\textcolor{Teal}{\comment{Henry: {#1}}}}
\newcommand{\todo}[1]{\textcolor{blue}{\comment{To Do: {#1}}}}
\newcommand{\clocode}[1]{{\texttt {#1}}}

\begin{document}
\title{Insert Title}
\date{\today}

\frame{\frametitle{Table of contents}\tableofcontents[currentsection]} 

\begin{frame}[fragile]
	\frametitle{introduction}

\end{frame}

\begin{frame}[fragile]
\frametitle{Overview of topic}
\end{frame}

\begin{frame}[fragile]
\frametitle{Intro to Clojure}
	\begin{itemize}
  	 \item Clojure is a dialect of Lisp
  	 \item First introduced in 2007 by Rich Hickey
  	 \item Immutable data structures
	 \item Built-in support for concurrency.
	 \item Runs on the Java Virtual Machine (JVM)
	\end{itemize}
\end{frame}

\begin{frame}[fragile]
\frametitle{Functional Languages}
	\begin{itemize}
	 \item Clojure is a functional language
  	 \item Treat computation as the evaluation of functions
  	 \item functional languages avoid direct memory manipulation
       \end{itemize}
\end{frame}

\begin{frame}[fragile]
\frametitle{Polish Prefix Notation}
Can be generalized to \clocode{(function arg1 arg2 ... argN)}.
	\begin{verbatim}
	(+ 2 3)
	=> 5
	\end{verbatim}
Basic function syntax: \clocode{(defn name [args] expr)}
	\begin{verbatim}
	(defn add1 [num] (+ num 1))
	(add1 3)
	=> 4
	\end{verbatim}
\end{frame}

\begin{frame}[fragile]
\frametitle{Vectors}
A type of collection in Clojure
	\begin{verbatim}
	(get [1 2 3 4 5] 3)	
	=> 4
	\end{verbatim}
\end{frame}

\begin{frame}[fragile]
\frametitle{High Order Functions}
%\hfcomment{mention code-as-data}
{\centering
 Functions can take functions as arguments\par
} 
\begin{verbatim}
(map add1 [0 1 2 3 4])
=>(1 2 3 4 5)
\end{verbatim}
\end{frame}

\begin{frame}[fragile]
\frametitle{Reduce}
known as fold in other Lisps
\begin{verbatim}
(reduce + [1 2 3])
=> 6
\end{verbatim}
\end{frame}

\begin{frame}[fragile]
\frametitle{Lazy Evaluation}
	\begin{itemize}
	  \item Delaying evaluation until the value is needed.
  	  \item Possible infinite sequences
  	  \item Has roots in Lambda Calculus
  	  \item Avoids the use of State as a programming tool.
       \end{itemize}
\end{frame}

\begin{frame}[fragile]
\frametitle{Pmap}
\end{frame}

\begin{frame}[fragile]
\frametitle{Reducers I}
\end{frame}

\begin{frame}[fragile]
\frametitle{Reducers II}
\end{frame}

\end{document}
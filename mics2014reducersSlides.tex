\documentclass{beamer}
\usetheme{Montpellier}
\usecolortheme{dolphin}

\usepackage{graphicx} %For jpg figure inclusion
\usepackage{times} %For typeface
\usepackage{epsfig}
\usepackage{color} %For Comments
\usepackage[all]{xy}
\usepackage{float}
\usepackage{subfigure} 
\usepackage{hyperref}
\usepackage{url}
\usepackage{parskip}
\usepackage{multirow}

%% Elena's favorite green (thanks, Fernando!)
\definecolor{ForestGreen}{RGB}{34,139,34}
%% Joe's Color.
\definecolor{JoesGold}{RGB}{204,102,0}
%% Henry's Color.
\definecolor{Teal}{RGB}{2,132,130}
% Uncomment this if you want to show work-in-progress comments
\newcommand{\comment}[1]{{\bf \tt  {#1}}}
% Uncomment this if you don't want to show comments
%\newcommand{\comment}[1]{}
\newcommand{\emcomment}[1]{\textcolor{ForestGreen}{\comment{Elena: {#1}}}}
\newcommand{\joecomment}[1]{\textcolor{JoesGold}{\comment{Joe: {#1}}}}
\newcommand{\hfcomment}[1]{\textcolor{Teal}{\comment{Henry: {#1}}}}
\newcommand{\todo}[1]{\textcolor{blue}{\comment{To Do: {#1}}}}
\newcommand{\clocode}[1]{{\texttt {#1}}}
\newcommand{\fast}[1]{\textcolor{green}{#1}}
\newcommand{\slow}[1]{\textcolor{red}{#1}}

\begin{document}
\title{Exploration of parallelization efficiency in the Clojure programming language}
\date{\today}

\begin{frame}
\frametitle{Exploration of parallelization efficiency in the Clojure programming language}
{\centering
Midwest Instruction and Computing Symposium\par
April 25, 2014\par
Henry Fellows, Joe Einertson, and Elena Machkasova\par
}
\end{frame}
%frame

\begin{frame}[fragile]
\frametitle{Introduction}
	Our project is a comparison of parallelism methods in the Clojure programming language.
	\begin{itemize}
	\item Relatively new language.
	\item Designed for efficent parallel operations.
	\item Recently added new parallel library.
	\end{itemize}
	Motivations.
	\begin{itemize}
	\item Interest in using Clojure as an educational tool.
	\item Using concurrency in functional language.
	\item Developing concurrent algorithms.
	\end{itemize}
	
\end{frame}
%frame

\begin{frame}
\frametitle{Table of contents}
\tableofcontents %[currentsection] I don't know why this doesn't work.} 
\end{frame}
%frame
\section{Overview of Clojure}

\begin{frame}[fragile]
\frametitle{Intro to Clojure}
	\begin{itemize}
  	 \item Clojure is a dialect of Lisp.
	 \item Runs on the Java Virtual Machine (JVM).
  	 \item First introduced in 2007 by Rich Hickey.
  	 \item Immutable data structures.
	 \item Built-in support for parallelism.
	 \end{itemize}
\end{frame}
%frame

\begin{frame}[fragile]
\frametitle{Functional Languages and Lisps}
	Functional Languages
	\begin{itemize}
	 \item Clojure is a functional language.
  	 \item Treat computation as the evaluation of functions.
  	 \item Functional languages avoid direct memory manipulation.
       \end{itemize}
	Lisp of a family of programming languages 
		\begin{itemize}
		 \item Lisp-1 
		 \item Common Lisp
		 \item Scheme
		 \item Racket
		 \item Clojure
		\end{itemize}
\end{frame}
%frame
\begin{frame}[fragile]
\frametitle{Prefix Notation}
Can be generalized to \clocode{(function arg1 ... argN)}.
	\begin{verbatim}
	(+ 2 3)
	=> 5
	\end{verbatim}
Basic function syntax: \clocode{(defn name [args] expr)}
	\begin{verbatim}
	(defn add1 [num] (+ num 1))
	(add1 3)
	=> 4
	\end{verbatim}
\end{frame}
%frame

\begin{frame}[fragile]
\frametitle{Vectors}
A type of collection in Clojure.
Accessing items by index is $O(\log n)$.
	\begin{verbatim}
	(get [2 7 4 9 5] 3)	
	=> 9
	\end{verbatim}
\end{frame}
%frame
\begin{frame}[fragile]
\frametitle{High Order Functions}
%\hfcomment{mention code-as-data}
 Functions can take functions as arguments.
	\begin{verbatim}
	(map add1 [0 1 2 3 4])
	 => [1 2 3 4 5]
	\end{verbatim}
Another high order function, \clocode{reduce}.
	\begin{verbatim}
	(reduce + [1 2 3])
	 => 6
	\end{verbatim}
The combination of \clocode{reduce} and \clocode{map}.
	\begin{verbatim}
	(reduce + (map sqrt [1 4 25]))
	 => 8
	\end{verbatim}
\end{frame}
%frame
%\begin{frame}[fragile]
%\frametitle{Lazy Evaluation}
%	\begin{itemize}
%	  \item Delaying evaluation until the value is needed.
%  	  \item Infinite sequences - so long as it is not all called.
%       \end{itemize}	
%	\begin{verbatim}
%	 (take 10 (range))
%	 => (0 1 2 3 4 5 6 7 8 9)
%	\end{verbatim}
%\end{frame}
%frame
\section{Clojure Concurrency}

\begin{frame}
\frametitle{Concurrency}
	\begin{itemize}
	 \item Most processors are now being built with multiple cores.
	 \item Concurrency is the execution of multiple computations simultaneously.
	 \item Programming concurrent programs is \textit{considered hard}.
	 \item Deadlocking: two tasks are waiting for resources that the other task holds.
	 \item Immutable data structures make concurrency easier.
	\end{itemize}	
\end{frame}
%frame
\begin{frame}
\frametitle{Parallel Computation in Clojure}
	Clojure has several methods of parallelism.
	\begin{itemize}
	\item \clocode{pmap} is one of the early methods of parallelism in Clojure.
	\item Reducers is a new library introduced in 2012.
	\end{itemize}
\end{frame}
%frame
\begin{frame}[fragile]
\frametitle{Pmap}
	\begin{itemize}
	 \item A parallel version of \clocode{map}.
	 \item Has the same syntax as \clocode{map}.
	 \item On a sufficiently large collection, it will create additional threads.
	\end{itemize}	
	\begin{verbatim} 
	(pmap add1 [0 1 2 3 4])
	=> [1 2 3 4 5]
	\end{verbatim}
\end{frame}
%frame
\begin{frame}[fragile]
\frametitle{Reducers}
	\begin{itemize}
	 \item Released by Rich Hickey in May 2012.
	 \item Built on Java's fork/join framework.
	 \item Reducers provides parallel higher-order functions, with the same names as their serial counterparts.
	 \item \clocode{r/fold} is used in place of reduce.
	\end{itemize}
\end{frame}
%frame
\begin{frame}[fragile]
\frametitle{Implementation of Reducers}
	\begin{itemize}
	\item All collections come with a traversal mechanism.
	\item All reducers functions (\clocode{r/map, r/filter}) except \clocode{r/fold} provide a recipe.
	\item \clocode{r/fold} causes the evaluation of all recipes attached to a collection in parallel.
	\item Fork/Join framework creates one thread per core (as reported by OS).
	\end{itemize}
	\begin{verbatim}
	(r/fold + (r/map sqrt [1 4 25]))
	 => 8
	\end{verbatim}
\end{frame}
%frame
\begin{frame}[fragile]
\frametitle{Test Structure}
	\begin{itemize}
	\item Computationally expensive operations on large sets of integers
	
	\end{itemize}
	Three tests:
	\begin{itemize}
	\item Count-primes
	\begin{verbatim}
	(reduce + (map (one-if-prime-else-zero [...])))
	\end{verbatim}
	\item Sum-primes
	\begin{verbatim}
	(reduce + (map (zero-if-composite-else-n [...])))
	\end{verbatim}
	\item Sum-sqrt
	\begin{verbatim}
	(reduce + (map (sqrt [...])))
	\end{verbatim}
	\end{itemize}
\end{frame}
%frame 

\begin{frame}
\frametitle{Test Sub-Structure}
\begin{table}
\begin{center}
\begin{tabular}{|l|l|}
\hline 
Name & Description \\
\hline
map + reduce & serial map, serial reduce \\
pmap + reduce & parallel map, serial reduce \\
map + r/fold & serial map, parallel reduce \\
pmap + r/fold & parallel map, parallel reduce\\
r/map + r/fold & reducers parallel map, parallel reduce\\
r/fold & parallel reduce\\
\hline
\end{tabular}
\end{center}
\caption{Configurations for our tests}\label{table:tests}
\end{table}
The \clocode{r/fold} configuration does not have a mapping phase: the test code was rewritten to make it work with a single reduce.
\end{frame}
%frame
\begin{frame}
\frametitle{Fermat Primality Test}
Two of the tests focus on prime testing;
	\begin{itemize}
	\item  Probabilistic algorithm called the \emph{Fermat primality test}.
	\item Two parameters: a number to test, and the number of trials to run.
	\item The more trials, the more accurate the result.
	\item We run it with five trials, and numbers on the order of one billion.
	\item \emph{Chosen because it is computationally expensive.}
	\end{itemize}
\end{frame}

\begin{frame}
\frametitle{Primes Tests}
	\begin{itemize}
	\item Counts number of primes in the given collection.
	\item Collection is 100,000 random integers between 0 and 1 billion.
	\item repeated 100 times, with new data each time. 
	\end{itemize}
Compare-sum-primes
	\begin{itemize}
	\item Sums the values of all primes in the given collection.
	\item Collection is 10,000 random integers  between 0 and 1 billion.
	\item repeated 1000 times, with new data each time. 
	\end{itemize}
\end{frame}
%frame

\begin{frame}
\frametitle{Compare-sum-sqrt}
	\begin{itemize}
	\item Sums the integer square root of each value in the given collection.
	\item Like prime testing, integer square root is computationally difficult and a good 'real world' example.
	\item Collection is 10,000 random integers uniformly distributed between 0 and 1 billion.
	\item repeated 1000 times, with new data each time. 
	\end{itemize}
\end{frame}
%frame

\begin{frame}
\frametitle{Test Enviroments}
\begin{itemize}
 \item an Intel i7 CPU, with 4 cores.
 \item an Intel i5 CPU, with 2 cores.
 \item an AMD FX-8350 CPU, with 8 cores.
\end{itemize}
\end{frame}
%frame

\begin{frame}
\frametitle{Sum-Primes Results}
\begin{table}
\begin{center}
\hspace*{-0.2in}  %%%%%%% to move the table left
\begin{tabular}{|l|c|c|c|c|c|c|}
\hline
Run & \multirow{2}{1.5cm}{reduce, \\ map} 
&  \multirow{2}{1.5cm}{reduce, \\ pmap} 
&  \multirow{2}{1.3cm}{r/fold,  \\ pmap} 
&   \multirow{2}{1.3cm}{r/fold, \\  map} 
& r/fold& \multirow{2}{1.3cm}{r/fold, \\  r/map}\\
& & & & &  & %%%% to force vertical lines in the second row
\\
\hline
i7 & 208.0 & \fast{66.4} & \fast{61.7} & 207.0 & \fast{57.2} &  \fast{54.6} \\
i5 & 279.3 & 250.6 & 284.3 & 280.8 & \fast{132.0} & \fast{131.0} \\
AMD & 266.9 & 225.1 & 248.4 & 275.5 & \fast{59.2} & \fast{63.6} \\
\hline
\end{tabular}
\end{center}
\caption{Sum-Primes averages (ms).}\label{table:sum-primes}
\end{table}
\end{frame}
%frame

\begin{frame}
\frametitle{Count-Primes Results}
\begin{table}[h!]
\begin{center}
\begin{tabular}{|l|c|c|c|c|c|}
\hline
Run & \multirow{2}{1.5cm}{reduce, \\ map} 
&  \multirow{2}{1.5cm}{reduce, \\ pmap} 
&  \multirow{2}{1.3cm}{r/fold,  \\ pmap} 
&   \multirow{2}{1.3cm}{r/fold, \\  map} 
& r/fold
\\
& & & & &   %%%% to force vertical lines in the second row
\\
\hline
i7  & 2084.6 & \fast{604.5} & \fast{597.1} & 2065.7 & \fast{535.8}\\
i5 & 2802.8 & 2567.7 & 2585.6 & 2774.0 & \fast{1269} \\
AMD & 2662.2 & 2411.3 & 2426.6 & 2647.9 & \fast{557.6}\\
\hline
\end{tabular}
\end{center}
\caption{Count-Primes averages (ms).}\label{table:count-primes}
\end{table}
\end{frame}
%frame

\begin{frame}
\frametitle{Sum-Sqrt Results}

\begin{table}[h!]
\begin{center}
\begin{tabular}{|l|c|c|c|c|c|}
\hline
Run & \multirow{2}{1.5cm}{reduce, \\ map} 
&  \multirow{2}{1.5cm}{reduce, \\ pmap} 
&  \multirow{2}{1.3cm}{r/fold,  \\ pmap} 
&   \multirow{2}{1.3cm}{r/fold, \\  map} 
& r/fold
\\
& & & & &   %%%% to force vertical lines in the second row
\\
\hline
i7 & 115.4 & 128.7 & 109.7 & \fast{28.6} & \fast{30.5}\\
i5 & 120.1 & \slow{401.3} & \slow{414.0} & \fast{60.0} & \fast{58.0} \\
AMD & 115.9 & \slow{359.5} & \slow{367.6} & \fast{32.8} & \fast{32.4}\\
\hline
\end{tabular}
\end{center}
\caption{Sum-Sqrt averages (ms).}\label{table:sum-sqrt}
\end{table}
\end{frame}
%frame

\section{Conclusion}
\begin{frame}
\frametitle{Pmap and Thread Thrashing}
	Pmap is unreliable.
	\begin{itemize}
	\item Running times ranging from close to the best parallel runs.
	\item To worse than serial methods.
	\item Up to 244\% slower than serial methods!
	\end{itemize}
Pmap creates too many threads.
	\begin{itemize}
	\item This causes \emph{thread thrashing}.
	\item The number of treads leads to excessive context switching.
	\item Causing the process to choke on its own overhead.
	\end{itemize}
\end{frame}
%frame

\begin{frame}
\frametitle{Reducers}
	\begin{itemize}
	\item Reducers is \emph{fast}, running 15\% faster than pmap, when pmap was working.
	\item \clocode{r/fold} + \clocode{r/map}, run as fast as the one step \clocode{r/fold}.
	\item Relatively reliable.
	\end{itemize}
\end{frame}
%frame

\begin{frame}
\frametitle{Environments}
Intel i7 
	\begin{itemize}
	\item 'Fastest' machine tested
	\item Resistant to thread thrashing.
	\item Caused by hyper-threading?
	\end{itemize}
Intel i5
	\begin{itemize}
	\item 'slowest' machine tested
	\item not resistant to thread thrashing.
	\end{itemize}
AMD Fx-8350
	\begin{itemize}
	\item Slightly resistant to thread thrashing.
	\item Does not scale as well.
	\item Due to micro-architecture?
	\end{itemize}
	
\end{frame}
%frame

\begin{frame}
\frametitle{Conclusion}
	There's a lot to look into;
	\begin{itemize}
	\item Thread balancing in reducers.
	\item Optimal thread management.
	\item The effects of CPU architecture on thread thrashing.
	\end{itemize}
We still want to continue on our main interest, parallel algorithm development in functional languages.
\break
The authors thank Jon Anthony for helpful discussions and methodology suggestions. 
\end{frame}

\end{document} 
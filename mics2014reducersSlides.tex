\documentclass{beamer}
\usetheme{Montpellier}
\usecolortheme{dolphin}

\usepackage{graphicx} %For jpg figure inclusion
\usepackage{times} %For typeface
\usepackage{epsfig}
\usepackage{color} %For Comments
\usepackage[all]{xy}
\usepackage{float}
\usepackage{subfigure} 
\usepackage{hyperref}
\usepackage{url}
\usepackage{parskip}

%% Elena's favorite green (thanks, Fernando!)
\definecolor{ForestGreen}{RGB}{34,139,34}
%% Joe's Color.
\definecolor{JoesGold}{RGB}{204,102,0}
%% Henry's Color.
\definecolor{Teal}{RGB}{2,132,130}
% Uncomment this if you want to show work-in-progress comments
\newcommand{\comment}[1]{{\bf \tt  {#1}}}
% Uncomment this if you don't want to show comments
%\newcommand{\comment}[1]{}
\newcommand{\emcomment}[1]{\textcolor{ForestGreen}{\comment{Elena: {#1}}}}
\newcommand{\joecomment}[1]{\textcolor{JoesGold}{\comment{Joe: {#1}}}}
\newcommand{\hfcomment}[1]{\textcolor{Teal}{\comment{Henry: {#1}}}}
\newcommand{\todo}[1]{\textcolor{blue}{\comment{To Do: {#1}}}}
\newcommand{\clocode}[1]{{\texttt {#1}}}

\begin{document}
\title{Exploration of parallelization efficiency in the Clojure programming language}
\date{\today}

\begin{frame}
\frametitle{Exploration of parallelization efficiency in the Clojure programming language}
{\centering
Midwest Instruction and Computing Symposium\par
April 25, 2014\par
Henry Fellows, Joe Einertson, and Elena Machkasova\par
}
\end{frame}

\frame{\frametitle{Table of contents}\tableofcontents[currentsection]} 

\begin{frame}[fragile]
\frametitle{Introduction}
\end{frame}

\begin{frame}[fragile]
\frametitle{Overview}
\end{frame}

\begin{frame}[fragile]
\frametitle{Intro to Clojure}
	\begin{itemize}
  	 \item Clojure is a dialect of Lisp
  	 \item First introduced in 2007 by Rich Hickey
  	 \item Immutable data structures
	 \item Built-in support for concurrency.
	 \item Runs on the Java Virtual Machine (JVM)
	\end{itemize}
\end{frame}

\begin{frame}[fragile]
\frametitle{Functional Languages}
	\begin{itemize}
	 \item Clojure is a functional language
  	 \item Treat computation as the evaluation of functions
  	 \item Functional languages avoid direct memory manipulation
       \end{itemize}
\end{frame}

\begin{frame}[fragile]
\frametitle{Polish Prefix Notation}
Can be generalized to \clocode{(function arg1 ... argN)}.
	\begin{verbatim}
	(+ 2 3)
	=> 5
	\end{verbatim}
Basic function syntax: \clocode{(defn name [args] expr)}
	\begin{verbatim}
	(defn add1 [num] (+ num 1))
	(add1 3)
	=> 4
	\end{verbatim}
\end{frame}

\begin{frame}[fragile]
\frametitle{Vectors}
A type of collection in Clojure
	\begin{verbatim}
	(get [1 2 3 4 5] 3)	
	=> 4
	\end{verbatim}
\end{frame}

\begin{frame}[fragile]
\frametitle{High Order Functions}
%\hfcomment{mention code-as-data}
 Functions can take functions as arguments
	\begin{verbatim}
	 (map add1 [0 1 2 3 4])
	 =>(1 2 3 4 5)
	\end{verbatim}
\end{frame}

\begin{frame}[fragile]
\frametitle{Reduce}
Also known as fold in other Lisps
	\begin{verbatim}
	 (reduce + [1 2 3])
	 => 6
	\end{verbatim}
\end{frame}

\begin{frame}[fragile]
\frametitle{Lazy Evaluation}
	\begin{itemize}
	  \item Delaying evaluation until the value is needed.
  	  \item infinite sequences - so long as it is not all called.
       \end{itemize}	
	\begin{verbatim}
	 (take 10 (range))
	 => (0 1 2 3 4 5 6 7 8 9)
	\end{verbatim}
\end{frame}

\begin{frame}
\frametitle{Concurrency}
	\begin{itemize}
	 \item Most processors are now being built with multiple cores.
	 \item Concurrency is the execution of multiple computations simultaneously.
	 \item Programming concurrent programs is \textit{hard}.
	 \item Deadlocking: two tasks are waiting for resources that the other task holds.
	\end{itemize}	
\end{frame}

\begin{frame}[fragile]
\frametitle{Pmap}
	\begin{itemize}
	 \item A parallel version of \clocode{map}
	 \item Has the same syntax as \clocode{map}.
	 \item \clocode{pmap} is lazy.
	 \item \clocode{doall} forces eager evaluation. 
	\end{itemize}	
	\begin{verbatim}
	(pmap add1 [0 1 2 3 4])
	=> (1 2 3 4 5)
	\end{verbatim}
\end{frame}

\begin{frame}[fragile]
\frametitle{Reducers I}
	\begin{itemize}
	 \item Released by Rich Hickey in May 2012.
	 \item Reducers provides parallel higher-order functions.
	 \item Mostly drop-in replacements.
	 \item Built on Java's fork/join framework.
	\end{itemize}	
\end{frame}

\begin{frame}[fragile]
\frametitle{Reducers II}
	\begin{itemize}
	 \item The names of reducers functions are the same as their serial counterparts
	 \item Except reduce, which gets called \clocode{fold}
	\end{itemize}	
	In our code, and in this presentation, we use the prefix \clocode{r/} to differentiate between the core versions and the reducers versions.
	
\end{frame}

\end{document}